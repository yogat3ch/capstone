\documentclass[10pt]{article}

\usepackage{fullpage}
\usepackage{setspace}
\usepackage{parskip}
\usepackage{titlesec}
\usepackage{xcolor}
\usepackage{lineno}





\PassOptionsToPackage{hyphens}{url}
\usepackage[colorlinks = true,
            linkcolor = blue,
            urlcolor  = blue,
            citecolor = blue,
            anchorcolor = blue]{hyperref}
\usepackage{etoolbox}
\makeatletter
\patchcmd\@combinedblfloats{\box\@outputbox}{\unvbox\@outputbox}{}{%
  \errmessage{\noexpand\@combinedblfloats could not be patched}%
}%
\makeatother


\usepackage{natbib}




\renewenvironment{abstract}
  {{\bfseries\noindent{\abstractname}\par\nobreak}\footnotesize}
  {\bigskip}

\renewenvironment{quote}
  {\begin{tabular}{|p{13cm}}}
  {\end{tabular}}

\titlespacing{\section}{0pt}{*3}{*1}
\titlespacing{\subsection}{0pt}{*2}{*0.5}
\titlespacing{\subsubsection}{0pt}{*1.5}{0pt}


\usepackage{authblk}


\usepackage{graphicx}
\usepackage[space]{grffile}
\usepackage{latexsym}
\usepackage{textcomp}
\usepackage{longtable}
\usepackage{tabulary}
\usepackage{booktabs,array,multirow}
\usepackage{amsfonts,amsmath,amssymb}
\providecommand\citet{\cite}
\providecommand\citep{\cite}
\providecommand\citealt{\cite}
% You can conditionalize code for latexml or normal latex using this.
\newif\iflatexml\latexmlfalse
\providecommand{\tightlist}{\setlength{\itemsep}{0pt}\setlength{\parskip}{0pt}}%

\AtBeginDocument{\DeclareGraphicsExtensions{.pdf,.PDF,.eps,.EPS,.png,.PNG,.tif,.TIF,.jpg,.JPG,.jpeg,.JPEG}}

\usepackage[utf8]{inputenc}
\usepackage[T2A]{fontenc}
\usepackage[ngerman,polish,english]{babel}








\begin{document}

\title{
\centering
A Meta-Analysis of the Effect of Mindfulness Training in Education on Self-Compassion}


\author[1]{Stephen Holsenbeck}%
\affil[1]{Northeastern University}%
\author[1]{Babatunde Aideyan}%
\author[1]{Mariya Shiyko PhD}%


\vspace{-1em}



  \date{\today}


\begingroup
\let\center\flushleft
\let\endcenter\endflushleft
\maketitle
\endgroup




\colorbox{teal!10}{%
\begin{minipage}{1\linewidth}%
\selectlanguage{english}
\begin{abstract}
\textbf{Introduction:~}The experience of the academic environment can be
competitive, stressful, and at times overwhelming for students and
faculty alike. Self-compassion (defined by~\hyperref[csl:1]{(Neff, Whittaker, \& Karl, 2017)} as
including self-kindness, common humanity, and mindfulness) has been
associated with general resourcefulness~\hyperref[csl:2]{(Martin, Kennett, \& Hopewell, 2019)},
self-regulation~\hyperref[csl:3]{(Dundas, Binder, Hansen, \& Stige, 2017)}, and well-being~\hyperref[csl:4]{(Gunnell, Mosewich, McEwen, Eklund, \& Crocker, 2017)} in
university students, but mindfulness practices evoking self-compassion
remain underutilized in academic contexts.~~\textbf{Methods:} This
meta-analysis evaluates three studies examining self-compassion as
measured by the Self-Compassion Scale~\hyperref[csl:1]{(Neff, Whittaker, \& Karl, 2017)} as a primary or
secondary outcome of mindfulness-based interventions conducted in
educational contexts. The study seeks to establish associations between
the characteristics of mindfulness-based interventions in educational
contexts, namely contact time, individual time (outside of the group
context), type of intervention (Mindfulness-based stress reduction or
Self-compassion course) and their influence on the~between-group
(treatment-control) mean standardized differences of scores on the
Self-compassion scale.~\textbf{Results:} The three included studies show
standardized mean difference effect sizes between treatment \& control
groups ranging from .57-.94. Total contact time had a nearly negligible
though significant~\(\beta=-0.0009,\ p\le.1\)correlation with
self-compassion~scores suggesting that shorter length courses may be
equally effective as longer courses in influencing self-compassion
outcomes. Intervention type,~specifically interventions between 2 \& 8
weeks modeled after the self-compassion training courses show a large
significant~effect~\(\beta = 0.35,\ p\leq.1\) on self-compassion outcome
measures.

\textbf{Conclusion:~} Two to~eight-week mindfulness courses embedded
within coursework can have a significant effect on self-compassion
scores for student cohorts.%
\end{abstract}%
\end{minipage}%
}




\colorbox{gray!10}{%
        \begin{minipage}{1\linewidth}%
            \vspace*{2pt}%Space before
\textbf{Keywords:} Self-compassion scale, SCS, SCS-SF, Mindfulness, Meditation, Education, Academia, University, Mindfulness-based Stress Reduction, MBSR, Yoga
            \vspace*{2pt}%Space after
        \end{minipage}%
        }


\section*{Introduction}

{\label{874460}}

Self-compassion, mindfulness and related interventions, are increasingly
receiving attention for their use in education and psychological
settings. Self-compassion is described as a~state of ``loving, connected
presence'' that is increasingly cited in the literature as being linked
to positive states of mind~\hyperref[csl:5]{(Neff \& McGehee, 2010)}~and overall
well-being~\hyperref[csl:6]{(Richards, Campenni, \& Muse-Burke, 2010)}. Mindfulness involves intentionally
developing an equanimous awareness of one's feelings, bodily sensations,
mental phenomena as well as external stimuli associated with the present
moment experience. Regular mindfulness training, through practices such
as meditation, supports the development of
attention~\hyperref[csl:7]{(Moore, Gruber, Derose, \& Malinowski, 2012)}~and mindful appreciation of one's daily
experience.~

Though the practice has received an increase of attention in research
settings in recent years, it has been investigated in clinical
psychology and psychiatry settings since the 1970s (Creswell, 2017).
Results from numerous randomized controlled trials investigating
mindfulness show efficacy for treating a broad range of psychological
symptoms. Mindfulness interventions, which include mindfulness-based
stress reduction (MBSR), mindfulness-based relapse prevention (MBRP),
and mindfulness-based cognitive therapy (MBCT) have shown significant
positive effects on depression and depression relapse, and resemble
results produced from antidepressant medication treatments (Creswell,
2017). Meta-analyses and RCTs also indicate that mindfulness
interventions produce significant positive effects for managing anxiety
(Strauss, Cavanagh, Oliver, \& Pettman, 2014; V\selectlanguage{ngerman}øllestad, Nielsen,
Nielsen, 2012), as well as treating addictive disorders (Creswell,
2017).

Mindfulness may offer significant utility in education settings. Young
students often experience a myriad of new emotions and feelings as they
progress through the school system and interact with teachers and
classmates on a regular basis. Students in higher education settings
typically deal with the stress of coursework requirements, as well as
environments of excessive substance use, and the impending doom of
student loan debt. Considering that many researchers have discussed what
they consider to be a ``mental health crisis'' in higher education, it
appears that innovative interventions such as mindfulness are necessary
in education settings (Francis \& Horn, 2017).

Mindfulness interventions have also shown positive effects in education
settings. A 2014 meta-analysis of mindfulness-based interventions
utilized in primary education settings produced a medium overall effect
size (g = 0.40; Zenner, Herrnleben-Kurz, and Walach, 2014). The authors
report a large effect (g = 0.80) size for studies investigating
mindfulness interventions and cognitive performance (Zenner,
Herrnleben-Kurz, and Walach, 2014), and small to medium effect sizes in
several other psychological variables. Based on the results from their
analysis, the researchers argue that ``mindfulness-based training in a
school context has effects that are seen mostly in the cognitive domain,
but also in psychological measures of stress, coping, and resilience''
(Zenner, Herrnleben-Kurz, and Walach, 2014, p. 16). No systematic
reviews or meta-analyses of mindfulness-based interventions across the
full spectrum of higher education exists, but proxy studies suggest a
robust evidence-basis for applying mindfulness interventions in higher
education. A meta-analysis that investigated outcomes of mindfulness in
a classroom setting and included a wide range of randomized controlled
intervention trials applied in higher education found a moderate effect
size for the subgroup of mindfulness interventions (g = 0.52; Davies,
Morriss, and Glazebrook, 2014). Additionally, a narrative review study
of mindfulness meditation concluded that it is a capable intervention
for helping college students manage stress and anxiety (Bamber \&
Schneider, 2016).

The primary goal of this study is to review and analyze research results
regarding the effect that mindfulness interventions have on
self-compassion in education settings. The construct of self-compassion
has advanced in recent decades with writings from Kristin Neff (2016a,
2016b). In the simplest terminology, self-compassion is the process of
directing compassion inwardly when dealing with difficult situations and
experiences. Neff summarizes the concept of self-compassion well in the
following quote; ``Instead of mercilessly judging and criticizing
yourself for various inadequacies or shortcomings, self-compassion means
you are kind and understanding when confronted with personal failings''
(Neff, 2016a, p. 1). As it pertains to mindfulness, Neff argues that the
practice is a key component of self-compassion (Neff, 2016a).
Self-compassion requires that individuals take note of their negative
feelings and emotions when dealing with difficult situations and viewing
such experiences objectively and with lucidity. For us to regard our
personal experiences with compassion, we must first be mindful of our
pain and suffering.

No systematic review or meta-analysis currently exists that examines a
self-compassion variable in education settings; however, several studies
indicate associations with positive outcomes. Gunnell et al. (2017)
discovered associations between well-being and self-compassion in
first-year college students and concluded that enhanced self-compassion
in the first year of college mitigates declines in well-being over the
course of their college career. Another study of 200 Pakistani
university students found that increased self-compassion significantly
predicts decreases in procrastination (Loona \& Khan, 2016). Finally,
\selectlanguage{polish}Ş\selectlanguage{english}enyuva et al. (2014) discovered positive correlations between self-compassion and emotional intelligence in a sample of 571 Turkish
nursing students. Together, these studies indicate that self-compassion
is an important factor in education settings.

A broad review of the literature has conjured two primary conclusions
that influenced the direction of this study; (1) mindfulness-based
interventions are capable of producing positive outcomes in education
settings; (2) self-compassion is meaningfully associated with positive
outcomes in education settings. Therefore, the major research questions
for this study is as follows; what effect do mindfulness-based
interventions applied in education settings have on the construct of
self-compassion, and how effective are such interventions on influencing
self-compassion?

\par\null\selectlanguage{english}
\begin{figure}[h!]
\begin{center}
\includegraphics[width=1.00\columnwidth]{figures/mass/mass}
\caption{{Pellentesque suscipit risus massa, non vestibulum libero euismod
feugiat. In hac habitasse platea dictumst. Maecenas rutrum lobortis
lobortis. Vestibulum convallis porttitor sem ac ultricies.
{\label{931769}}%
}}
\end{center}
\end{figure}

\par\null

Mauris nec massa leo. Mauris ac diam auctor nisl imperdiet porta. Sed
sit amet neque eget nisi dictum placerat. Duis sit amet pellentesque
odio. Cras scelerisque sem a consectetur vehicula. Aliquam interdum
luctus fringilla. Nunc sollicitudin, lorem in semper viverra, dui nisi
sodales sem, ut condimentum erat leo eget arcu. Donec pharetra aliquam
metus, non pulvinar tellus interdum a. Mauris a ante pharetra, mollis
enim in, eleifend erat. Pellentesque suscipit risus massa, non
vestibulum libero euismod feugiat. In hac habitasse platea dictumst.
Maecenas rutrum lobortis lobortis. Vestibulum convallis porttitor sem ac
ultricies. Mauris volutpat fringilla nisl blandit semper. Proin nec
iaculis sem. Aenean neque ipsum, pretium a faucibus non, tincidunt ut
sapien.

Nunc a aliquet sem, eget aliquet purus. Vestibulum ac placerat mauris.
Proin sed dolor ac justo semper iaculis. Donec varius, nibh sit amet
finibus tristique, sapien ante interdum odio, et pretium sapien libero
nec massa. In hac habitasse platea dictumst. Donec vel augue ac sapien
imperdiet pretium. Maecenas gravida risus id ultricies dignissim.
Maecenas gravida felis quis dolor faucibus, sed maximus lorem tristique.
Nam hendrerit quam quis ante porta posuere. Fusce finibus maximus orci
at porttitor. Nulla tempor ex a porttitor consequat. Quisque quis tempor
eros. Donec nisi mauris, sollicitudin in dapibus eu, interdum ultricies
quam Fig {\ref{931769}}.

\section*{Section}

{\label{507127}}

Nunc a aliquet sem, eget aliquet purus. Vestibulum ac placerat mauris.
Proin sed dolor ac justo semper iaculis. Donec varius, nibh sit amet
finibus tristique, sapien ante interdum odio, et pretium sapien libero
nec massa. In hac habitasse platea dictumst. Donec vel augue ac sapien
imperdiet pretium. Maecenas gravida risus id ultricies dignissim.
Maecenas gravida felis quis dolor faucibus, sed maximus lorem
tristique~\(e^{i\pi}+1=0\)

\section*{Acknowledgements}

{\label{687807}}

Lorem ipsum dolor sit amet, consectetur adipiscing elit. Cras egestas
auctor molestie. In hac habitasse platea dictumst. Duis turpis tellus,
scelerisque sit amet lectus ut, ultricies cursus enim. Integer fringilla
a elit at fringilla. Lorem ipsum dolor sit amet, consectetur adipiscing
elit. Nulla congue consequat consectetur. Duis ac mi ultricies, mollis
ipsum nec, porta est.

\selectlanguage{english}
\clearpage
\section*{References}\sloppy

\phantomsection
\label{csl:3}Dundas, I., Binder, P.-E., Hansen, T. G. B., \& Stige, S. H. (2017). {Does a short self-compassion intervention for students increase healthy self-regulation? {A} randomized control trial}. \textit{Scandinavian Journal of Psychology}, \textit{58}(5), 443–450. \url{https://doi.org/10.1111/sjop.12385}

\phantomsection
\label{csl:4}Gunnell, K. E., Mosewich, A. D., McEwen, C. E., Eklund, R. C., \& Crocker, P. R. E. (2017). {Don't be so hard on yourself! {Changes} in self-compassion during the first year of university are associated with changes in well-being}. \textit{Personality and Individual Differences}, \textit{107}, 43–48. \url{https://doi.org/10.1016/j.paid.2016.11.032}

\phantomsection
\label{csl:2}Martin, R. D., Kennett, D. J., \& Hopewell, N. M. (2019). {Examining the importance of academic-specific self-compassion in the academic self-control model}. \textit{The Journal of Social Psychology}, 1–16. \url{https://doi.org/10.1080/00224545.2018.1555128}

\phantomsection
\label{csl:7}Moore, A., Gruber, T., Derose, J., \& Malinowski, P. (2012). {Regular, brief mindfulness meditation practice improves electrophysiological markers of attentional control}. \textit{Frontiers in Human Neuroscience}, \textit{6}. \url{https://doi.org/10.3389/fnhum.2012.00018}

\phantomsection
\label{csl:5}Neff, K. D., \& McGehee, P. (2010). {Self-compassion and Psychological Resilience Among Adolescents and Young Adults}. \textit{Self and Identity}, \textit{9}(3), 225–240. \url{https://doi.org/10.1080/15298860902979307}

\phantomsection
\label{csl:1}Neff, K. D., Whittaker, T. A., \& Karl, A. (2017). {Examining the {Factor} {Structure} of the {Self}-{Compassion} {Scale} in {Four} {Distinct} {Populations}: {Is} the {Use} of a {Total} {Scale} {Score} {Justified}?}. \textit{Journal of Personality Assessment}, \textit{99}(6), 596–607. \url{https://doi.org/10.1080/00223891.2016.1269334}

\phantomsection
\label{csl:6}Richards, K., Campenni, C., \& Muse-Burke, J. (2010). {Self-care and Well-being in Mental Health Professionals: The Mediating Effects of Self-awareness and Mindfulness}. \textit{Journal of Mental Health Counseling}, \textit{32}(3), 247–264. \url{https://doi.org/10.17744/mehc.32.3.0n31v88304423806}


\end{document}

